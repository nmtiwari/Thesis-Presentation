\section*{Summary}
\subsection{Related Work}
\begin{frame}
  \begin{itemize}
    \item RAMCloud, Piccolo, GraphLab, parallelDBs
    	\begin{itemize}
    	  \item Fine-grained writes requires replication for fault tolerance
    	\end{itemize}
    \item Pregel, iterative MapReduce
    	\begin{itemize}
    	  \item Specialized models, can't perform interactive mining
    	\end{itemize}
    \item DryadLINQ, FlumeJava
    	\begin{itemize}
    	  \item Similar to RDD, but cannot share datasets efficiently across
    	  queries
    	\end{itemize}
    \item Nectar
    	\begin{itemize}
    	  \item Automatic expression caching over distributed FS
    	\end{itemize}
    \item PacMan
    	\begin{itemize}
    	  \item Memeory cache for HDFS with writes over network/disk
    	\end{itemize}
  \end{itemize}
\end{frame}

\subsection{Errors and Checking Mechanisms}
\begin{frame}
Possible Errors in RDD based programming,
\begin{itemize}
  \item Errors due to malformed or corrupt data
  	\begin{itemize}
  	  \item Use \texttt{filter} transformation to discard bad inputs
  	  \item \texttt{map}, if it is possible to fix the bad input
  	  \item \texttt{flatmap}, try fixing the input but fall back to discarding
  	\end{itemize}
  \item Job aborted due to stage failure: Task not serializable
  	\begin{itemize}
  	  \item happens in the process of serializing the closure before sending it
  	  to workers, fail if the object is not serializable
  	  \item cause: intialize a variable on the driver (master), but then try to
  	  use it on one of the workers
  	\end{itemize}
  \item UnsupportedOperation exception
  	\begin{itemize}
  	  \item Spark uses Scala's static type system and inference technique,
  	  however there are reports on exceptions triggered due to invalid operations
  	\end{itemize}
  \item OutOfMemory errors
  	\begin{itemize}
  	  \item caused by caching large reusable data
  	\end{itemize}
\end{itemize}
\end{frame}

\begin{frame}
Some works,
\begin{itemize}
  \item Spores and Silo
  	\begin{itemize}
  	  \item Potential hazards when using closures incorrectly are: 
  	  \item memory leaks
  	  \item race conditions due to capturing mutable references
  	  \item runtime serialization errors due to unintended capture of references
  	  \item Spores: well-behaved closures with controlled environments that can
  	  avoid various hazards
  	  \item Silo: Distributed Programming via Safe Closure Passing (uses spores
  	  in syntactic and type based restrictions)
  	\end{itemize}
  \item A Characteristic Study on Out of Memory Errors in Distributed
  Data-Parallel Applications
\end{itemize}
\end{frame}

\subsection{Conclusion}
\begin{frame}
1) RDDs are simple and efficient programming model for a broad range of
applications

2) Key to performance: it allows applications to avoid costly
disk accesses by reliably storing the data in memory.

3) Spark implementation of RDDs,
\begin{itemize}
  \item APIs in Scala, Java, Python and R
  \item Integrates HDFS, Amazon S3, Hive, HBase, Cassandra, etc.
  \item Can run on clusters managed by Hadoop YARN, Apache Mesos, also
  standalone 
  \item High level library support, SparkSQL, Spark Streaming, MLlib (machine
  learning), GraphX (graph), R, etc.
\end{itemize}
\end{frame}

\subsection{References}
\begin{frame}
\scriptsize{
[1] M. Zaharia, M. Chowdhury, T. Das, A. Dave, J. Ma, M. McCauley, M.J.
Franklin, S. Shenker, I. Stoica. Resilient Distributed Datasets: A
Fault-Tolerant Abstraction for In-Memory Cluster Computing, NSDI 2012, April
2012
}
\newline

\scriptsize{
[2] Talk: Resilient Distributed Datasets: A Fault-Tolerant Abstraction for
In-Memory Cluster Computing, by M. Zaharia, NSDI 2012, San Jose, CA, April 2012.
}
\newline

\scriptsize{
[3] Talk: A Deeper Understanding of Spark Internals, by Aaron Davidson
(Databricks), Spark Summit 2014, San Francisco, CA, June 2014
}
\newline

\scriptsize{
[4] Spark Programming Guide 1.6.0, URL:
http://spark.apache.org/docs/latest/programming-guide.html
}
\end{frame}

\subsection{Thank you}
\begin{frame}
\begin{beamercolorbox}[center]{white}
  {\Large Questions?}

  \vspace{2em}\hfill

  \url{http://www.cs.iastate.edu/~ganeshau/}
\end{beamercolorbox}
\end{frame}
