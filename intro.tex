\section{Background}
\subsection{MSR}
    \begin{frame}
    \frametitle{Mining Software Repository (MSR)}
    Analysis of rich data available in software repositories such as version control, bug
    tracking, source code, mailing list and organiation information.

    \begin{itemize}
        \item Significant contribution in Programming language design and Software engineering domains.
        \item Impact of MSR tools can be increased if the process of building and widely distributing MSR tools is eased.
        \footnote{Robles et.al, Replicating MSR:A study of the potential replicability of papers
        published in the Mining Software Repositories proceedings, 2010}
    \end{itemize}
    \end{frame}

\section{Acknowledgement}
\subsection{Acknowledgement}
        \begin{frame}
        \frametitle{Acknowledgement}
%        \centering
            \begin{itemize}
               \item \tiny{\emph{Nitin Tiwari, Ganesha Upadhyaya, and Hridesh Rajan, "Candoia: a
                platform and ecosystem for mining software repositories tools.",
                ICSE'16}} (Distinguished Poster Award)\\
                \item \tiny{\emph{Tiwari, Nitin M., Ganesha Upadhyaya, Dr. Hoan Anh Nguyen and Hridesh Rajan.
                    "Candoia: A Platform for Building and Sharing Mining Software Repositories Tools as Apps", MSR'17}} \\
                \item This work was supported in part by the US National Science Foundation under grants CCF-15-18897, CNS-15-13263, and CCF-14-23370.
                \item Would like to thank Dalton D. Mills and Trey Erenberger for helping with
                Candoia frontend implementation, Eric Lin for implementing several Candoia apps
                and Dr. Robert Dyer for valuable feedback on publication draft
            \end{itemize}
        \end{frame}


\section{Overview}
    \begin{frame}
        \begin{itemize}
            \item Problem
            \begin{itemize}
                \item Building easily customizable, adoptable and applicable mining software repository tools
            \end{itemize}

            \item Solution
            \begin{itemize}
                \item An ecosystem which offers suitable abstractions and computational means to realize the process for building and sharing MSR tools as apps.
            \end{itemize}

        %  \item Candoia Eco-system
        %  	  \begin{itemize}
        %	    \item Available abstractions
        %	    \item Process of building MSR tools
        %	  \end{itemize}

            \item Evaluation
        % \begin{itemize}
        %      \item Available abstractions
        %      \item Process of building MSR tools
        %   \end{itemize}

            \item Related works, Conclusion, \& Future Work
              \begin{itemize}
                \item Existing open source tools and frameworks
                \item Open source datasets
              \end{itemize}
        \end{itemize}
    \end{frame}
